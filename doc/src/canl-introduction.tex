%
%% Copyright (c) Members of the EGEE Collaboration. 2004-2010.
%% See http://www.eu-egee.org/partners for details on the copyright holders.
%% 
%% Licensed under the Apache License, Version 2.0 (the "License");
%% you may not use this file except in compliance with the License.
%% You may obtain a copy of the License at
%% 
%%     http://www.apache.org/licenses/LICENSE-2.0
%% 
%% Unless required by applicable law or agreed to in writing, software
%% distributed under the License is distributed on an "AS IS" BASIS,
%% WITHOUT WARRANTIES OR CONDITIONS OF ANY KIND, either express or implied.
%% See the License for the specific language governing permissions and
%% limitations under the License.
%
% -*- mode: latex -*-

\section{Introduction}

This document serves as a developer's guide and could be
seen as an API reference too, even though comments in
the header files may give the reader better insights into that matter.

The \CANL API can be devided by functionality into two parts:

\begin{itemize}
\item \textit{\CANL Main API} is used to establish (secure) client-server
connection with one or both sides authenticated, send or receive data,
\item \textit{\CANL Certificate API}
\end{itemize}
The Main API is independent from the Certificate API, but the latter
one shares a number of data types and functions with the first one.

\subsection{Language Bindings}
\CANL is developed in C language as well as C++ and Java language bindings,
however this document covers only the C interface.

\subsection{Getting and Building Library}
TODO package names

external dependencies:
\begin{itemize}
\item c-ares -- asynchronous resolver library
\item openssl -- cryptography and SSL/TLS toolkit
\end{itemize}

\subsection{General Guidelines}

\marginpar{Naming conventions}%
All function names are prefixed with canl\_

\marginpar{Input and output arguments}%
All structures and objects passed in output of functions
(even though pointers are used as a help)
are dynamically allocated, so proper functions to free the allocated 
memory has to be called. e.g. \verb'canl_free_ctx()' deallocates members
of the stucture.

\marginpar{Opaque types}%
Almost all types used in caNl are \textit{Opaque types}- i.e. their structure is 
not exposed to users. To use and/or modify these structures API call has 
to be used. Example of opaque type is \verb'canl_ctx'.

\marginpar{Return values}%
The return type of most of the API functions is \verb'canl_err_code' which
 in most cases can be interpreted as int. Unless specified otherwise, zero
return value means success, non-zero failure. Standard error codes from 
\verb'errno.h' are used as much as possible.
TODO openssl mapping

Few API function return \verb'char *'. In such a~case
\verb'NULL' indicates an error, non-null value means success.

\subsection{Context and Parameter Settings}
\label{s:context}
All the API functions use a \emph{context} parameter of type \verb'canl_ctx' 
to maintain state information like error message and code. 
Some API functions also use an \emph{io context} of type \verb'canl_io_handler'
which keeps information about each particular connection 
(\eg socket number, oid, SSL context).The caller can create as many 
contexts as needed, all of them will be independent. When calling
\verb'canl_create_ctx()' or \verb'canl_create_io_handler()' all members 
of the objects are initialized with default values which are often 
NULL for pointer type and 0 in case of int and similar types.

\section{\CANL Components}
\label{s:common}

\subsection{Header Files}

Header files for the common structures and functions are summarized in
table~\ref{t:cheaders}.

\begin{table}[h]
\begin{tabularx}{\textwidth}{>{\tt}lX}
canl.h & Definition of context objects and \textit{Main API} common 
functions declarations. \\
canl\_ssl.h & Declaration of functions that use X509 certificates 
based authentication mechanism (pretty much dependent on 
openssl library funcions).\\
canl\_cred.h & Definition of context objects of the 
\textit{Certificate API}  and  functions declarations.\\
\end{tabularx}
\caption{Header files}
\label{t:cheaders}
\end{table}

\subsection{Building Client Programs}
The easiest way to build programs using \CANL in C is to use
GNU's libtool to take care of all the dependencies:
\begin{verbatim}
libtool --mode=compile gcc -c example1.c -D_GNU_SOURCE
libtool --mode=link gcc -o example1 example1.o -lcanl_c
\end{verbatim}

\subsection{Context}
\label{s:canl_ctx}
\marginpar{Context initialization}%
There are two opaque data structures representing caNl \textit{Main API} 
context: \verb'canl_ctx' and \verb'canl\_io\_handler' (see 
section~\ref{s:context}.
\verb'canl_ctx' must be initialized before any caNl API call.
\verb'canl_io_handler' must be initialized before function call 
representing io operation (e.g. \verb'canl_io_connect()') and after
\verb'canl_ctx' initialization. 
\begin{lstlisting}
#include <canl.h>
#include <canl_ssl.h>

canl_io_handler my_io_h = NULL;
canl_ctx my_ctx;
my_ctx = canl_create_ctx();
err = canl_create_io_handler(my_ctx, &my_io_h);
\end{lstlisting}
There is one opaque data structure representing \CANL 
\textit{Certificate API}: \verb'canl_cred'.
It must only be initialized before function calls
that use this context as a parametr.
\begin{lstlisting}
#include <canl.h>
#include <canl_cred.h>

canl_ctx ctx = NULL;
ctx = canl_create_ctx();
\end{lstlisting}
\marginpar{Obtaining error description}%
\verb'canl_ctx' stores details of all errors which has occurred since 
context initialization, in human readable format. To obtain it use 
\verb'canl_get_error_message()':
\begin{lstlisting}
printf("%s\n", canl_get_error_message(my_ctx));
\end{lstlisting}

\marginpar{Context deallocation}%
It is recommended to free the memory allocated to each 
context if they are not needed anymore, in first case \verb'canl_io_handler'
, then \verb'canl_ctx' in case of the \texit{Main API}:
\begin{lstlisting}
if (my_io_h)
	canl_io_destroy(my_ctx, my_io_h);
canl_free_ctx(my_ctx);
\end{lstlisting}
as for the Certificate API:
\begin{lstlisting}
canl_cred_free(ctx, signer);
\end{lstlisting}
