%
%% Copyright (c) Members of the EGEE Collaboration. 2004-2010.
%% See http://www.eu-egee.org/partners for details on the copyright holders.
%% 
%% Licensed under the Apache License, Version 2.0 (the "License");
%% you may not use this file except in compliance with the License.
%% You may obtain a copy of the License at
%% 
%%     http://www.apache.org/licenses/LICENSE-2.0
%% 
%% Unless required by applicable law or agreed to in writing, software
%% distributed under the License is distributed on an "AS IS" BASIS,
%% WITHOUT WARRANTIES OR CONDITIONS OF ANY KIND, either express or implied.
%% See the License for the specific language governing permissions and
%% limitations under the License.
%
% -*- mode: latex -*-

\section{Client-Server Authenticated Connection}
\label{s:cs-auth-conn}

For client-server authenticated connection we just use \CANL
\textit{Main API} calls. In time of writing this paper 
\CANL use \textit{openssl - SSL/TLS and cryptography toolkit}. 
However, core of the \CANL has been developed to be as independent
 on any cryptography toolkit as possible, so it may support 
other libraries in the future.

\subsection{Main API Reference guide}
These are the functions of the \verb'Main API' that do not 
use \textit{openssl API} calls or variable types directly
 (as a parameter or in their definitions):

\begin{itemize}
\item \verb'canl_ctx canl_create_ctx()'--This function
returns an initialized authentication context object
\item \verb'void CANL_CALLCONV canl_free_ctx(canl_ctx cc)'--This
function will free the authentication context, releasing 
all associated information.  The context must not be used after this call.
\begin{itemize}
\item param cc - the authentication context to free.
\end{itemize}
\item \verb'canl_err_code 
canl_create_io_handler(canl_ctx cc, canl_io_handler *io)' -- 
This function will create an \verb'i/o handler' from the authentication 
context. This handler shall be passed to all I/O-related functions.
\begin{itemize}
\item param cc - the authentication context
\item param io -  return an initialized \verb'i/o context', or NULL if 
it did not succeed.
\item return - \CANL error code
\end{itemize}

\item \verb'canl_err_code canl_io_close(canl_ctx cc, canl_io_handler io)' --
This function will close an existing connection.  The 'io' object may 
be reused by another connection. It is safe to call this 
function on an io object which was connected.
\begin{itemize}
\item param ac - The authentication context
\item param io - The \verb'i/o context'
\item return - \verb'canl_err_code'
\end{itemize}
\item \verb'canl_err_code canl_io_connect(canl_ctx cc, canl_io_handler io, 
const char *host, const char *service, int port, gss_OID_set auth_mechs, 
int flags, struct timeval *timeout)' --
This function will try to connect to a server object, 
doing authentication (if not forbidden)
\begin{itemize}
\item param ac - The authentication context
\item param io - The \verb'i/o context'
\item param host - The server to which to connect
\item param service - TODO DK
\item param port - The port on which the server is listening
\item param auth\_mechs - Authentication mechanism to use
\item flags - TODO
\item param timeout - The timeout after which to drop the connect attempt
\item return - \verb'canl_err_code'
\end{itemize}
\item \verb'canl_io_accept(canl_ctx cc, canl_io_handler io, 
int fd, struct sockaddr s_addr, int flags, 
canl_principal *peer, struct timeval *timeout)' -- This function will 
setup a server to accept connections from clients, doing 
authentication (if not forbidden)
\begin{itemize}
\item param ac - The authentication context
\item param io - The \verb'i/o context'
\item param fd - \TODO
\item param port - The port on which the server is listening
\end{itemize}

\end{itemize}

\subsection{Secure Client-Server Connection Example}
We give example of caNl client, for complete sample see 
\verb'canl_samples_server.c' in 
\TODO Where to find sources (package?)- Frantisek?

Include nesessary header files:
\begin{lstlisting}
#include <canl.h>
#include <canl_ssl.h>
\end{lstlisting}

Initialize context and set parameters:
\begin{lstlisting}
canl_ctx my_ctx;
canl_io_handler my_io_h = NULL;
my_ctx = canl_create_ctx();
err = canl_create_io_handler(my_ctx, &my_io_h);
err = canl_ctx_set_ssl_cred(my_ctx, serv_cert, serv_key, NULL, NULL);
\end{lstlisting}

Connect to the server, send something then read the response:

\begin{lstlisting}
err = canl_io_connect(my_ctx, my_io_h, p_server, NULL, port, NULL, 0, &timeout);
if (err) {
        printf("[CLIENT] connection to %s cannot be established: %s\n",
               p_server, canl_get_error_message(my_ctx));
        goto end;
    }
err = canl_io_write (my_ctx, my_io_h, buf, buf_len, &timeout);
if (err <= 0) {
        printf("can't write using ssl: %s\n",
               canl_get_error_message(my_ctx));
        goto end;
    }
err = canl_io_read (my_ctx, my_io_h, buf, sizeof(buf)-1, &timeout);
if (err > 0) {
        buf[err] = '\0';
        printf ("[CLIENT] received: %s\n", buf);
        err = 0;
    }
\end{lstlisting}

Free the allocated memory:

\begin{lstlisting}
if (my_io_h)
        canl_io_destroy(my_ctx, my_io_h);
canl_free_ctx(my_ctx);
\end{lstlisting}
